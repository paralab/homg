\documentclass[12pt]{article}
\renewcommand*\familydefault{\sfdefault}
%\usepackage{amssymb,amsmath,amsfonts,comment}
%\usepackage{amsmath,amssymb,graphicx,subfigure,psfrag}
\usepackage{amsmath,amssymb,graphicx,subfigure,psfrag}
\usepackage{amssymb,mathrsfs}
\usepackage[margin=1in]{geometry}
\usepackage{graphicx}
\usepackage{color,pdfcolmk}
\newcommand{\todo}[1]{\noindent\emph{\textcolor{red}{Todo: #1\:}}}
\newcommand{\note}[1]{\noindent\emph{\textcolor{green}{Note: #1\:}}}
\newcommand{\referee}[1]{\vspace{3ex}\noindent{\textcolor{blue}{#1}}\\[2ex]}
\newcommand{\nnn}{\mathbf{n}}
\newcommand{\fff}{\mathbf{f}}
\newcommand{\uuu}{\mathbf{u}}
\title{Reply to reviewer 1}

\begin{document}

\subsection*{Reply to Referee 2:}

Thanks for the careful reading and the helpful comments.
Please find below point-by-point replies to your comments and
questions. To give you an overview of all the changes in the paper, we
also provide a diff-document that highlights the changes between the
initial submission and this resubmission.

\subsubsection*{Detailed replies (your comments in blue)}

\referee{ Comments to the Author\\[.2ex] The paper presents a very
  nice computational survey of the most natural (and popular)
  approaches for the application of geometric Multigrid to second
  order elliptic problems discretized with high-order finite
  elements. This is timely and interesting study, given the increasing
  importance of high-order methods on modern computing
  architectures. The authors are to be commended for the interesting
  choice of problems (high-order meshes, polynomial orders up to 16,
  etc.) as well as for the freely available companion Matlab
  software. I liked the presentation and I found the results
  interesting.}
\noindent
Thanks.

\referee{ I therefore recommend the manuscript for publication with
  the following minor remarks:}

\referee{ 1) Section 2 refers to Figure 1 (first paragraph on page 4),
  but discusses only the first two approaches depicted there. This can
  be confusing.}
\noindent 
We have changed the figure reference to focus on the first two approaches. 

\referee{ 2) Why are the h-Multigrid spaces nested for the domains in
  Figure 4?}
\noindent
We use a reference mesh for basis functions of polynomial degree $p$,
as well as for restriction and prolongation. These polynomial basis
functions are then mapped to the actual geometry (on which they might
not be polynomials of order $p$ anymore). On the reference domain, the
finite element spaces are nested.

%\todo{I've to think about this a bit more; I know that this is how we
%  do things, but it's not described in the paper at all.}

\referee{ 3) It looks like SSOR is amplifying some error components
  when used as a smoother in Figure 3 (e.g. for p=4). Isn't Ak SPD in
  this case?}
\noindent
Yes, the matrix is symmetric and positive definite. However,
convergence proofs of Jacobi and Gauss-Seidel smoothers that show that
every components is decreased usually require stronger assumptions,
such as diagonal dominance, which is satisfied for first-order
discretizations, but not for high-order FEM-based discretizations.

\referee{ 4) Have you tried using the $\ell_1$-Jacobi smoother from [2]?
  This smoother is always convergent, so it may approve the results of
  e.g. Table II.}
\noindent
We have added the implementation and results for the $\ell_1$-Jacobi smoother. Indeed,
having a always convergent smoother is reassuring, but as the results indicate, it 
was not always very efficient. It was interesting to observe that in many cases using
point-wise jacobi as a smoother with multigrid as a preconditioner for CG worked better
than using the $\ell_1$-smoother.

\referee{ 5) What is the coarse grid in Section 3.2.2, the one
  obtained by h-coarsening (and interpolation)?}
\noindent
Yes, we have clarified this in the text as well.

\referee{ 6) A few typos: "loose" on page 1, "corresponds" on page 4,
  extra "rate" on page 7.}
\noindent
Thanks---these have been corrected.


\bibliographystyle{unsrt}
\bibliography{../ccgo,../mg}


\end{document}


