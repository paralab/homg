\documentclass[12pt]{article}
\renewcommand*\familydefault{\sfdefault}
%\usepackage{amssymb,amsmath,amsfonts,comment}
%\usepackage{amsmath,amssymb,graphicx,subfigure,psfrag}
\usepackage{amsmath,amssymb,graphicx,subfigure,psfrag}
\usepackage{amssymb,mathrsfs}
\usepackage[margin=1in]{geometry}
\usepackage{graphicx}
\usepackage{color,pdfcolmk}
\newcommand{\todo}[1]{\noindent\emph{\textcolor{red}{Todo: #1\:}}}
\newcommand{\note}[1]{\noindent\emph{\textcolor{blue}{Note: #1\:}}}
\newcommand{\referee}[1]{\vspace{3ex}\noindent{\textcolor{blue}{#1}}\\[2ex]}
\newcommand{\nnn}{\mathbf{n}}
\newcommand{\fff}{\mathbf{f}}
\newcommand{\uuu}{\mathbf{u}}
\title{Reply to reviewer 1}

\begin{document}

\subsection*{Reply to Referee 1:}

Thanks for the careful reading and the helpful comments.
Please find below point-by-point replies to your comments and
questions. To give you an overview of all the changes in the paper, we
also provide a diff-document that highlights the changes between the
initial submission and this resubmission.

\subsubsection*{Detailed replies (your comments in blue)}

\referee{
Summary impression:\\[.2ex]
Given the interest in high-order methods and the notable cost of the
underlying solver for poisson-type problems in addition to the variety
of options, the core thesis of the manuscript is well-placed.  Yet,
the presentation of the numerics does not lead the reader to conclude
anything from the comparison --- ie, there are too many ways to
interpret the convergence based on iterations and the cost of the
methods.  The primary concern as stated in the Contributions section
was to address the "actual performance", but as it stands the results
do not attempt to answer this.  Indeed, actual timings are avoided
(and clearly stated this way), the performance model is not realistic
(also as stated, since it can vary from implementation and
architecture), and the iterations are not combined with the perceived
cost in a clear way.  Again, the targeted contributions of this
manuscript would be quite interesting, but the numerics and discussion
could be significantly improved to support many of the statements.}
\noindent
our text here

\referee{
Non-Minor Concerns\\[2ex]
It's likely that h, p, or structured low-order MG are preferred to a
pure AMG method, but the restricition to "discontinuous or anisotropic
coefficients, nor consider ill-shaped elements" on page 3 should
really consider the implications of this.  The three methods will be
challenged in this setting whereas AMG has been used effectively.}
\noindent

\referee{ There is a consistent claim that matrix assembly for
  high-order should be avoided, but some context or citation is needed
  to support this (very) general remark.  If the coefficients and
  elements vary widely then the cost of reconstruction may benefit
  from a matrix assembly, particularly if it is repeatedly used.  From
  the results it's not clear at all that a SpMV based on an assembled
  matrix is *not* the most efficient action.  In the end was AMG
  tested?  THe last line appears that it was: "When combined with
  algebraic multigrid for the low-order operator, the smoother on the
  finest mesh can either use the low-order or the high-order
  residual. Initial numerical tests indicate that the latter choice is
  advantageous, but this should be studied more systematically."
  Perhaps this is referring to some side tests; it is suggested then
  to support these statements more.  Also, the code appears to use the
  assembled matrix (cf.  grid.m assemblepoisson()), so perhaps this
  referee is just confused about this point.  Is the matrix assembled
  or is everything matrix free?  And how is the cost compared?}

\referee{ Figure 2 needs some work.  Column 1 displays the smoothing
  properties.  But why should we expect Jacobi, Chebyshev, and SSOR to
  display classic smoothing properties for high-order problems?  And
  why does weighted Jacobi anihilate all high-order modes (for
  linears) in the top corner plot?  In addition, the use of a single
  coarse grid is used here (so it should be labeled two-grid rather
  than V-cycle, but this is minor), but what is the coarse grid?  For
  presentation, a suggestion is to move the rather lengthy paragraph
  in the Figure 2 caption to the the section where it is cited.
}
\noindent
...

\referee{
To use Cheychev, are the Arnoldi iterations considered in the cost?}

\referee{ Algorithm 4.1 requires $M$ to be symmetric.  Are the
  smoothers applied symmetrically?  SSOR does not appear to do this in
  particular.}
\noindent
Yes!

\referee{ Section 4 gets very long without reporting anything on
  relative efficiency.  The iteration counts show that the methods are
  converging in a reasonable fashion.  But " As a consequence, the
  iteration numbers reported in the next section can be used to
  compare the efficiency of the different methods." on page 12 is as
  close as it gets to comparing the total work of the method(s).  The
  introduction reports this as a major contribution and so it should
  be central in section 4.
}

\referee{
A Few Minor Issues\\[.3ex]
page 3: discretization $\rightarrow$ discretizations}

\referee{
page 3: on the performance $\rightarrow$ in the performance }

\referee{
page 3: The self references to 15-17 are unnecesary as presented.  If
the intent is to motivate high-order, then other references should be
used.  This happens again on page 5 in section 2.3 where [15-16] are
mentioned with little connection to the current work.}

\referee{
page 10: "number of multigrid v-cycles or of CG iterations" needs a
rewording
}

\referee{ page 12: Table I should list these values as iterations
  counts somewhere (and there is a hanging sentence below the table
  from the itemize on the previous page) }

\noindent

\end{document}
